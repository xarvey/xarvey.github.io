%%%%%%%%%%%%%%%%%%%%%%%%%%%%%%%%%%%%%%%%%
% Medium Length Professional CV
% LaTeX Template
% Version 2.0 (8/5/13)
%
% This template has been downloaded from:
% http://www.LaTeXTemplates.com
%
% Original author:
% Trey Hunner (http://www.treyhunner.com/)
%
% Important note:
% This template requires the resume.cls file to be in the same directory as the
% .tex file. The resume.cls file provides the resume style used for structuring the
% document.
%
%%%%%%%%%%%%%%%%%%%%%%%%%%%%%%%%%%%%%%%%%

%----------------------------------------------------------------------------------------
%	PACKAGES AND OTHER DOCUMENT CONFIGURATIONS
%----------------------------------------------------------------------------------------

\documentclass{resume} % Use the custom resume.cls style

\usepackage[left=0.5in,top=0.5in,right=0.5in,bottom=0.5in]{geometry} % Document margins
\usepackage{hyperref}

\name{Yao Xiao} % Your name
\address{430 Wood Street Room 929\\ West Lafayette , Indiana 47906} % Your address
\address{(765)~$\cdot$~409~$\cdot$~3418 \\ xiao67@purdue.edu} % Your phone number and email




\begin{document}
\center
{\url{http://xarvey.github.io/}}
%----------------------------------------------------------------------------------------
%	EDUCATION SECTION
%----------------------------------------------------------------------------------------

\begin{rSection}{Objective}
To obtain an internship position for Twitter
%To obtain an internship in the computer science field in summer 2016 or winter 2015, utilizing my skills in software engineering and machine intelligence.
\end{rSection}
\begin{rSection}{Education}

{\bf Purdue University, West Lafayette} \hfill {\em December 2016} \\ 
Bachelor of Science in Computer Science \\
Overall GPA: 3.57/4.00

\end{rSection}

%----------------------------------------------------------------------------------------
%	WORK EXPERIENCE SECTION
%----------------------------------------------------------------------------------------
\begin{rSection}{Technical Strengths}

\begin{tabular}{ @{} >{\bfseries}l @{\hspace{6ex}} l }
Computer Languages & Familiar with C, C++, Java, Python, Bash. Knows Delphi, R, ARM, X86\\
Web Designing & HTML, mysql, Javascript, CSS, Apache,mongodb \\
Tools & Git, Vim, Emacs, Sublime,Latex
\end{tabular}

\end{rSection}

\begin{rSection}{Experience}

\begin{rSubsection}{Software Engineering Intern }{May 2015 - August 2015}{Homeaway.com}{Austin,TX}
\item Working on Core UI team, used Java and Javascript to create a proctor endpoint for A/B testing 
\item Refactored the current code to separate the logic layer and presentation layer

\end{rSubsection}

\begin{rSubsection}{Teaching Assistant}{January 2013 - December 2014}{Purdue University}{West Lafayette, IN}
\item Designed class assignments on topics of random forest in CS240 (Programming in C) and a chart maker CS 177(Programming in Python)
\item Wrote assignments and documentation for labs and projects as well as testing script in bash and python for grading.
%\item Hold help sessions to help students with their coding and debugging
\end{rSubsection}

%\begin{rSubsection}{Research Assistant}{May 2014 - July 2014}{}{}
%\item Set up fijiVM on Raspberry Pi for faculty members
%\item Providing a testing platform for algorithm verification 
%\end{rSubsection}

%------------------------------------------------
\end{rSection}


%----------------------------------------------------------------------------------------
%	EXAMPLE SECTION
%----------------------------------------------------------------------------------------

\begin{rSection}{Projects}

\begin{rSubsection}{A Javascript compiler}{Spring 2015}{Purdue University}{West Lafayette,IN}
\item Using Lex/Yacc(front end) and LLVM C++(back end) to implement a mini javascript compiler
\item Generated diagnosis report of a statement using the dynamic backward program slice method  
\item Optimized code using common sub-expression elimination.
\end{rSubsection}


\begin{rSubsection}{Find a room}{Fall 2014}{Purdue University}{West Lafayette,IN}
\item Created an indoor navigator app for users finding destination indoor.
\item Focused on backend design and the navigation algorithm
\item Using HTML,CSS,Javascript, mongodb, and meteor JS framework to create the app
\end{rSubsection}

%\begin{rSubsection}{Some Homeaway shit}{Summer 2015}%{Homeaway.com}{Austin,TX}
%\item shit
%\end{rSubsection}

%\begin{rSubsection}{PeteTwitt}{Spring 2014}{}{}
%\item Collaborated with fellow classmates to design a pseudo-twitter, supporting user posting message
%\item Allowing user subscribing others and get there messages
%\item Created a search engine by various metric
%\end{rSubsection}

%\begin{rSubsection}{Purdue Menu Master}{Spring 2014}{}{}
%\item Developed a web based application for students getting the menu for dining court
%\item Incorporated a review system to help students choosing the dining court to have meal.
%\end{rSubsection}



%Section content\ldots

\end{rSection}

%----------------------------------------------------------------------------------------
%	TECHNICAL STRENGTHS SECTION
%----------------------------------------------------------------------------------------
\begin{rSection}{Honors}{}{}


Dean of student \&\& Semester Honor  \hspace{90mm} Fall  2014-Present\\ 
Purdue CS Continuing Student Scholarship Recipient \\
Participating in ACM-ICPC East-Central North America Regional, earned 25th place(2nd among Purdue) 



\end{rSection}

%----------------------------------------------------------------------------------------

\end{document}
