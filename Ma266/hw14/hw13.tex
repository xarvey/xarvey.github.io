\documentclass{article}

\usepackage{fancyhdr}
\usepackage{extramarks}
\usepackage{amsmath}
\usepackage{amsthm}
\usepackage{amsfonts}
\usepackage{tikz}
\usepackage[plain]{algorithm}
\usepackage{algpseudocode}

\usetikzlibrary{automata,positioning}

%
% Basic Document Settings
%

\topmargin=-0.45in
\evensidemargin=0in
\oddsidemargin=0in
\textwidth=6.5in
\textheight=9.0in
\headsep=0.25in

\linespread{2.0}

\pagestyle{fancy}
\lhead{\hmwkAuthorName}
\chead{\hmwkClass\ (\hmwkClassInstructor\ \hmwkClassTime)}
\rhead{\firstxmark}
\lfoot{\lastxmark}
\cfoot{\thepage}

\renewcommand\headrulewidth{0.4pt}
\renewcommand\footrulewidth{0.4pt}

\setlength\parindent{0pt}

%
% Create Problem Sections
%

\newcommand{\enterProblemHeader}[1]{
    \nobreak\extramarks{}{Problem \arabic{#1} continued on next page\ldots}\nobreak{}
    \nobreak\extramarks{Problem \arabic{#1} (continued)}{Problem \arabic{#1} continued on next page\ldots}\nobreak{}
}

\newcommand{\exitProblemHeader}[1]{
    \nobreak\extramarks{Problem \arabic{#1} (continued)}{Problem \arabic{#1} continued on next page\ldots}\nobreak{}
    \stepcounter{#1}
    \nobreak\extramarks{Problem \arabic{#1}}{}\nobreak{}
}

\setcounter{secnumdepth}{0}
\newcounter{partCounter}
\newcounter{homeworkProblemCounter}
\setcounter{homeworkProblemCounter}{1}
\nobreak\extramarks{Problem \arabic{homeworkProblemCounter}}{}\nobreak{}

%
% Homework Problem Environment
%
% This environment takes an optional argument. When given, it will adjust the
% problem counter. This is useful for when the problems given for your
% assignment aren't sequential. See the last 3 problems of this template for an
% example.
%
\newenvironment{homeworkProblem}[1][-1]{
    \ifnum#1>0
        \setcounter{homeworkProblemCounter}{#1}
    \fi
    \section{Problem \arabic{homeworkProblemCounter}}
    \setcounter{partCounter}{1}
    \enterProblemHeader{homeworkProblemCounter}
}{
    \exitProblemHeader{homeworkProblemCounter}
}

%
% Homework Details
%   - Title
%   - Due date
%   - Class
%   - Section/Time
%   - Instructor
%   - Author
%

\newcommand{\hmwkTitle}{Homework\ \#13}
\newcommand{\hmwkDueDate}{Dec 11th, 2015}
\newcommand{\hmwkClass}{Differential Equation}
\newcommand{\hmwkClassTime}{Section 061}
\newcommand{\hmwkClassInstructor}{Professor Heather Lee}
\newcommand{\hmwkAuthorName}{Yao Xiao}
\newcommand{\La}{\mathcal{L}}

%
% Title Page
%

\title{
    \vspace{2in}
    \textmd{\textbf{\hmwkClass:\ \hmwkTitle}}\\
    \normalsize\vspace{0.1in}\small{Due\ on\ \hmwkDueDate\ at 3:10pm}\\
    \vspace{0.1in}\large{\textit{\hmwkClassInstructor\ \hmwkClassTime}}
    \vspace{3in}
}

\author{\textbf{\hmwkAuthorName}}
\date{}

\renewcommand{\part}[1]{\textbf{\large Part \Alph{partCounter}}\stepcounter{partCounter}\\}

%
% Various Helper Commands
%

% Useful for algorithms
\newcommand{\alg}[1]{\textsc{\bfseries \footnotesize #1}}

% For derivatives
\newcommand{\deriv}[1]{\frac{\mathrm{d}}{\mathrm{d}x} (#1)}

% For partial derivatives
\newcommand{\pderiv}[2]{\frac{\partial}{\partial #1} (#2)}

% Integral dx
\newcommand{\dx}{\mathrm{d}x}

% Alias for the Solution section header
\newcommand{\solution}{\textbf{\large Solution}}

% Probability commands: Expectation, Variance, Covariance, Bias
\newcommand{\E}{\mathrm{E}}
\newcommand{\Var}{\mathrm{Var}}
\newcommand{\Cov}{\mathrm{Cov}}
\newcommand{\Bias}{\mathrm{Bias}}

\begin{document}

\maketitle

\pagebreak


\begin{homeworkProblem}
\subsection{O}
\[
\begin{split}
	x'=x+y \\
	y' = 4x + y
\end{split}
\]
Apply Laplace on the both sides.

The first equation becomes

\[
	sX(s)-0=X(s)+Y(s)
\]
\[
	(s-1)X(s)=Y(s)
\]

The second one becomes
\[
sY(s)-2=4X(s)+Y(s)
\]
 \[
 (s-1)Y(s)=4X(s)+2
 \]
 We combine these to equations, we get
 \[
 X(s)=\frac{1}{2}[\frac{1}{s-3}-\frac{1}{s+1}]
 \]
 Find the inverse of laplace, we get
 \[
 	x(t)=\frac{1}{2}(e^{3t}-e^{-t})
 \]
 Also
 \[
 	x'(t)=\frac{1}{2}(3e^{3t}-e^{-t})
 \]
 \[
 y=x'-x=e^{3t}+e^{-t}
 \]
\end{homeworkProblem}

\begin{homeworkProblem}
\subsection{P}
\subsubsection{(a)}
  \[
\begin{split}
  \begin{bmatrix}
    1 & 0 \\
    2 & -3
  \end{bmatrix} \\
  \end{split}
  \]
  We find that the eigenvalue are
 \[ \gamma_1=1 \ \ \gamma_2=-3 \]
  and the corresponding eigenvectors are
  \[
  v_1 = \begin{bmatrix}
	2\\	
	1\\
	\end{bmatrix}
  \] and
  \[
  v_2 = \begin{bmatrix}
	0\\	
	1\\
	\end{bmatrix}
  \] 
  So
  \[
  x=c_1\begin{bmatrix}
	2\\	
	1\\
	\end{bmatrix}e^t+c_2 \begin{bmatrix}
	0\\	
	1\\
	\end{bmatrix}e^{-3t}
  \]
  The particular solution will be
  \[
  X_p=\begin{bmatrix}
	a_1e^{2t}+a_2\\	
	b_1e^{2t}+b_2
	\\
	\end{bmatrix}
  \]
  And 
  \[
  X'_p=\begin{bmatrix}
	2a_1e^{2t}\\	
	2b_1e^{2t}
	\\
	\end{bmatrix}
  \]
  Plug it in the X,we get
  \[
  \begin{bmatrix}
	2a_1e^{2t}\\	
	2b_1e^{2t}
	\\
	\end{bmatrix} =
	  \begin{bmatrix}
    1 & 0 \\
    2 & -3
  \end{bmatrix}
	\begin{bmatrix}
	a_1e^{2t}+a_2\\	
	b_1e^{2t}+b_2
	\end{bmatrix}
	+
	\begin{bmatrix}
	5e^{2t} \\
	3
	\end{bmatrix}
  \]
  Solve it, plug it in, we get
  \[
  \begin{split}
  a_1=5 \\
  a_2=0 \\
  b_1=2 \\
  b_2=1
\end{split}  
  \]
  So \[x_p=
  \begin{bmatrix}
	5e^{2t} \\
	5e^{2t}+1
	\end{bmatrix}
  \]
   \[
  x=c_1\begin{bmatrix}
	2\\	
	1\\
	\end{bmatrix}e^t+c_2 \begin{bmatrix}
	0\\	
	1\\
	\end{bmatrix}e^{-3t}+
	 \begin{bmatrix}
	5e^{2t} \\
	5e^{2t}+1
	\end{bmatrix}
  \]
  \subsubsection{(b)}


  We get 
  \[
  x=c_1\begin{bmatrix}
	2\\	
	1\\
	\end{bmatrix}e^t+c_2 \begin{bmatrix}
	0\\	
	1\\
	\end{bmatrix}e^{-3t}+x_p(t)
  \]
  
  And \[
  x_p(t)=  \begin{bmatrix}
	Acos(t)+Bsin(t)\\	
	Ccos(t)+Dsin(t)\\
	\end{bmatrix}
 \]
 and
 \[
  x'_p(t)=  \begin{bmatrix}
	-Asin(t)+Bcos(t)\\	
	-Csin(t)+Dcos(t)\\
	\end{bmatrix}
 \]
 Plug it in to the original equation, solve it. We get
 \[
 x_p(t)= \begin{bmatrix}
	-5cos(t)+5sin(t)\\	
	-4cos(t)+2sin(t)\\
	\end{bmatrix}
 \]
 \[
   x=c_1\begin{bmatrix}
	2\\	
	1\\
	\end{bmatrix}e^t+c_2 \begin{bmatrix}
	0\\	
	1\\
	\end{bmatrix}e^{-3t}+ \begin{bmatrix}
	-5cos(t)+5sin(t)\\	
	-4cos(t)+2sin(t)\\
	\end{bmatrix}
 \]
\end{homeworkProblem}



\end{document}
