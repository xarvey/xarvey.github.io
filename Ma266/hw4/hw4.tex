\documentclass{article}

\usepackage{fancyhdr}
\usepackage{extramarks}
\usepackage{amsmath}
\usepackage{amsthm}
\usepackage{amsfonts}
\usepackage{tikz}
\usepackage[plain]{algorithm}
\usepackage{algpseudocode}

\usetikzlibrary{automata,positioning}

%
% Basic Document Settings
%

\topmargin=-0.45in
\evensidemargin=0in
\oddsidemargin=0in
\textwidth=6.5in
\textheight=9.0in
\headsep=0.25in

\linespread{2.0}

\pagestyle{fancy}
\lhead{\hmwkAuthorName}
\chead{\hmwkClass\ (\hmwkClassInstructor\ \hmwkClassTime)}
\rhead{\firstxmark}
\lfoot{\lastxmark}
\cfoot{\thepage}

\renewcommand\headrulewidth{0.4pt}
\renewcommand\footrulewidth{0.4pt}

\setlength\parindent{0pt}

%
% Create Problem Sections
%

\newcommand{\enterProblemHeader}[1]{
    \nobreak\extramarks{}{Problem \arabic{#1} continued on next page\ldots}\nobreak{}
    \nobreak\extramarks{Problem \arabic{#1} (continued)}{Problem \arabic{#1} continued on next page\ldots}\nobreak{}
}

\newcommand{\exitProblemHeader}[1]{
    \nobreak\extramarks{Problem \arabic{#1} (continued)}{Problem \arabic{#1} continued on next page\ldots}\nobreak{}
    \stepcounter{#1}
    \nobreak\extramarks{Problem \arabic{#1}}{}\nobreak{}
}

\setcounter{secnumdepth}{0}
\newcounter{partCounter}
\newcounter{homeworkProblemCounter}
\setcounter{homeworkProblemCounter}{1}
\nobreak\extramarks{Problem \arabic{homeworkProblemCounter}}{}\nobreak{}

%
% Homework Problem Environment
%
% This environment takes an optional argument. When given, it will adjust the
% problem counter. This is useful for when the problems given for your
% assignment aren't sequential. See the last 3 problems of this template for an
% example.
%
\newenvironment{homeworkProblem}[1][-1]{
    \ifnum#1>0
        \setcounter{homeworkProblemCounter}{#1}
    \fi
    \section{Problem \arabic{homeworkProblemCounter}}
    \setcounter{partCounter}{1}
    \enterProblemHeader{homeworkProblemCounter}
}{
    \exitProblemHeader{homeworkProblemCounter}
}

%
% Homework Details
%   - Title
%   - Due date
%   - Class
%   - Section/Time
%   - Instructor
%   - Author
%

\newcommand{\hmwkTitle}{Homework\ \#4}
\newcommand{\hmwkDueDate}{September 19th, 2015}
\newcommand{\hmwkClass}{Differential Equation}
\newcommand{\hmwkClassTime}{Section 061}
\newcommand{\hmwkClassInstructor}{Professor Heather Lee}
\newcommand{\hmwkAuthorName}{Yao Xiao}

%
% Title Page
%

\title{
    \vspace{2in}
    \textmd{\textbf{\hmwkClass:\ \hmwkTitle}}\\
    \normalsize\vspace{0.1in}\small{Due\ on\ \hmwkDueDate\ at 3:10pm}\\
    \vspace{0.1in}\large{\textit{\hmwkClassInstructor\ \hmwkClassTime}}
    \vspace{3in}
}

\author{\textbf{\hmwkAuthorName}}
\date{}

\renewcommand{\part}[1]{\textbf{\large Part \Alph{partCounter}}\stepcounter{partCounter}\\}

%
% Various Helper Commands
%

% Useful for algorithms
\newcommand{\alg}[1]{\textsc{\bfseries \footnotesize #1}}

% For derivatives
\newcommand{\deriv}[1]{\frac{\mathrm{d}}{\mathrm{d}x} (#1)}

% For partial derivatives
\newcommand{\pderiv}[2]{\frac{\partial}{\partial #1} (#2)}

% Integral dx
\newcommand{\dx}{\mathrm{d}x}

% Alias for the Solution section header
\newcommand{\solution}{\textbf{\large Solution}}

% Probability commands: Expectation, Variance, Covariance, Bias
\newcommand{\E}{\mathrm{E}}
\newcommand{\Var}{\mathrm{Var}}
\newcommand{\Cov}{\mathrm{Cov}}
\newcommand{\Bias}{\mathrm{Bias}}

\begin{document}

\maketitle

\pagebreak

\begin{homeworkProblem}
16.
The statement of the problem implies 
\[
\frac{dT}{dt}=k(T-70)
\]

We also know that \(T(0) = 200\) and \(T(1) = 190 \)

So we can get 
\[T = \frac{\int-70ke^{-kt}dt}{e^{-kt}} = 70 + Ce^{kt} \]

Since \( T(0)=200 \)  so \( C=130\)

 \( T=70+130e^{kt} \), when \( t= 1\) we plug it in, we could get \( 70+130e^k=190 \)
\( k=ln\sqrt{12/13}\)

\(70+130e^{ln(12/13)t}=150\)

 \(t \approx 6.065 \)
\end{homeworkProblem}

\begin{homeworkProblem}
21.
\[
\begin{split}
F=ma \\
m\frac{dv}{dt}=-mg-v/30 \\
\frac{dv}{dt	}+\frac{v}{4.5}=-g \\
\end{split}
\]
Solve it, plug in t=0, v=20 get:
\( v=64.1e^{-t/4.5}-44.1\)
So distance 

\(x=-44.1t-299.45*e^{-0.222t}+C \)
plug in t=0, x=30,
we get C=318.5.

so \(x=-44.1t-299.45e^{-0.22t}+318.5 \)

when \( v=0 \) it reaches the max height.
we get \( t=1.683s \)
\[ x=45.783m \]

b) when the ball is dropping. We use similar method which could get(don't wanna type the entire procedure in latex = =)
\[
	x=4.5g(t+4.5e^{-0.22t})-4.5^2 
\]
we could get t=3.446, and therefore the entire time is 1.683+3.446=5.1129s
 

\end{homeworkProblem}

\begin{homeworkProblem}
2.4, 17:
\[ y'=ty(3-y) \]
\[\partial f/\partial = 3t-2ty \]
\[ dy/dt =  t(3y-y^2) \]
\[ 1/3(ln(y)-ln(3-y)=\frac{t^2}{2} + C \]
\[ y(t)=\frac{3e^{3t^2/2}}{3e/y_0-3+3e^{3t^2/2}} \]

\end{homeworkProblem}

\begin{homeworkProblem}
Supplementary problem D:
\[ 
\begin{split}
\frac{dy}{dt}=y^2-4y, y(0)=8 \\
\frac{dy}{y^2-4y}=dt \\
\frac{ln(4-y)-ln(y)}{4} =t + C \\
\frac{(ln(-4)-ln(8))}{4}=C \\
\frac{ln(4-y)-ln(y)}{4} =t + \frac{(ln(-4)-ln(8))}{4} \\
\ln(4-y)-ln(y)=4t+ln(-2) \\
y=\frac{4}{1-(e^{4t}/2)}
\end{split}
\]

since we want the solution to be continuous, we could get \( y\neq \frac{ln(2)}{4} \) , but the solution need to be containing 0, so the answer is 
\[ (-\infty	, \frac{ln(2)}{4})  \]


\end{homeworkProblem}


\end{document}
