\documentclass{article}

\usepackage{fancyhdr}
\usepackage{extramarks}
\usepackage{amsmath}
\usepackage{amsthm}
\usepackage{amsfonts}
\usepackage{tikz}
\usepackage[plain]{algorithm}
\usepackage{algpseudocode}

\usetikzlibrary{automata,positioning}

%
% Basic Document Settings
%

\topmargin=-0.45in
\evensidemargin=0in
\oddsidemargin=0in
\textwidth=6.5in
\textheight=9.0in
\headsep=0.25in

\linespread{2.0}

\pagestyle{fancy}
\lhead{\hmwkAuthorName}
\chead{\hmwkClass\ (\hmwkClassInstructor\ \hmwkClassTime)}
\rhead{\firstxmark}
\lfoot{\lastxmark}
\cfoot{\thepage}

\renewcommand\headrulewidth{0.4pt}
\renewcommand\footrulewidth{0.4pt}

\setlength\parindent{0pt}

%
% Create Problem Sections
%

\newcommand{\enterProblemHeader}[1]{
    \nobreak\extramarks{}{Problem \arabic{#1} continued on next page\ldots}\nobreak{}
    \nobreak\extramarks{Problem \arabic{#1} (continued)}{Problem \arabic{#1} continued on next page\ldots}\nobreak{}
}

\newcommand{\exitProblemHeader}[1]{
    \nobreak\extramarks{Problem \arabic{#1} (continued)}{Problem \arabic{#1} continued on next page\ldots}\nobreak{}
    \stepcounter{#1}
    \nobreak\extramarks{Problem \arabic{#1}}{}\nobreak{}
}

\setcounter{secnumdepth}{0}
\newcounter{partCounter}
\newcounter{homeworkProblemCounter}
\setcounter{homeworkProblemCounter}{1}
\nobreak\extramarks{Problem \arabic{homeworkProblemCounter}}{}\nobreak{}

%
% Homework Problem Environment
%
% This environment takes an optional argument. When given, it will adjust the
% problem counter. This is useful for when the problems given for your
% assignment aren't sequential. See the last 3 problems of this template for an
% example.
%
\newenvironment{homeworkProblem}[1][-1]{
    \ifnum#1>0
        \setcounter{homeworkProblemCounter}{#1}
    \fi
    \section{Problem \arabic{homeworkProblemCounter}}
    \setcounter{partCounter}{1}
    \enterProblemHeader{homeworkProblemCounter}
}{
    \exitProblemHeader{homeworkProblemCounter}
}

%
% Homework Details
%   - Title
%   - Due date
%   - Class
%   - Section/Time
%   - Instructor
%   - Author
%

\newcommand{\hmwkTitle}{Homework\ \#11}
\newcommand{\hmwkDueDate}{November 23th, 2015}
\newcommand{\hmwkClass}{Differential Equation}
\newcommand{\hmwkClassTime}{Section 061}
\newcommand{\hmwkClassInstructor}{Professor Heather Lee}
\newcommand{\hmwkAuthorName}{Yao Xiao}
\newcommand{\La}{\mathcal{L}}

%
% Title Page
%

\title{
    \vspace{2in}
    \textmd{\textbf{\hmwkClass:\ \hmwkTitle}}\\
    \normalsize\vspace{0.1in}\small{Due\ on\ \hmwkDueDate\ at 3:10pm}\\
    \vspace{0.1in}\large{\textit{\hmwkClassInstructor\ \hmwkClassTime}}
    \vspace{3in}
}

\author{\textbf{\hmwkAuthorName}}
\date{}

\renewcommand{\part}[1]{\textbf{\large Part \Alph{partCounter}}\stepcounter{partCounter}\\}

%
% Various Helper Commands
%

% Useful for algorithms
\newcommand{\alg}[1]{\textsc{\bfseries \footnotesize #1}}

% For derivatives
\newcommand{\deriv}[1]{\frac{\mathrm{d}}{\mathrm{d}x} (#1)}

% For partial derivatives
\newcommand{\pderiv}[2]{\frac{\partial}{\partial #1} (#2)}

% Integral dx
\newcommand{\dx}{\mathrm{d}x}

% Alias for the Solution section header
\newcommand{\solution}{\textbf{\large Solution}}

% Probability commands: Expectation, Variance, Covariance, Bias
\newcommand{\E}{\mathrm{E}}
\newcommand{\Var}{\mathrm{Var}}
\newcommand{\Cov}{\mathrm{Cov}}
\newcommand{\Bias}{\mathrm{Bias}}

\begin{document}

\maketitle

\pagebreak


\begin{homeworkProblem}
\subsection{6.6 - 4}
\[
\int^t_0 (t-\tau)^2 cos(2\tau)d\tau 
\]
So we could let \(g(t)=t^2\)
\(h(t)=cos(2t)\) \\
So \( G(s) = \frac{2}{s^3} \) \\ 
\( H(s) = \frac{s}{s^2+4} \)

\[
G(s)H(s)=\frac{2s}{s^3(s^2+4)}
\]
\end{homeworkProblem}

\begin{homeworkProblem}
\subsection{6.6 -5}
\[
\begin{split}
\int^t_0 e^{-(t-\tau)}sin(\tau)d\tau\\
g(t)=e^{-t}\\
h(t)=sin(t) \\
G(s)=\frac{1}{s+1}\\
H(s)=\frac{1}{s^2+1}
\end{split}
\]
So the result should be 
\[
G(s)H(s)=\frac{1}{(s+1)(s^2+1)}
\]
\end{homeworkProblem}

\begin{homeworkProblem}
\subsection{7.2-22}
\[
  \begin{bmatrix}
    3 & -2 \\
    2 & -2
  \end{bmatrix}
  *
  \begin{bmatrix}
    4e^{2t} \\
    2e^{2t}
  \end{bmatrix} =
    \begin{bmatrix}
    8e^{2t} \\
    4e^{2t}
  \end{bmatrix}
  = \begin{bmatrix}
    (4e^{2t})' \\
    (2e^{2t})'
  \end{bmatrix}
  \] 
  So it satisfies the condition
\end{homeworkProblem}

\begin{homeworkProblem}
\subsection{7.2-23}
Plug x into formula 1, we get
\[
  x'=\begin{bmatrix}
     e^t(3+2t) \\
    2e^t(1+t) 
  \end{bmatrix}
  \]
  which is equal to
  \[
 x'= \begin{bmatrix}
  3e^t+2te^t \\
  2e^t+2te^t
  \end{bmatrix}
  \] 
  So it satisfies the condition
\end{homeworkProblem}

\begin{homeworkProblem}
\subsection{M}
For Tank 1:
RateIn = 2oz/gal * 5gal/min = 10oz/min\\
RateOut = x1oz/50gal*5gal/min=x1/10 oz/min

Hence, the solution is 
\[
	x1'(t)=10-\frac{x1(t)}{10}
\]

For Tank 2: \\

RateIn = x1/10 \\
RateOut =  x2/20 * 55= x2/4 \\

Hence, the solution is 
\[
	x2'(t)=\frac{x1}{10}-\frac{x2}{4}
\]

\end{homeworkProblem}


\end{document}
