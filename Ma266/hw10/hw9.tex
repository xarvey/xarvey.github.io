\documentclass{article}

\usepackage{fancyhdr}
\usepackage{extramarks}
\usepackage{amsmath}
\usepackage{amsthm}
\usepackage{amsfonts}
\usepackage{tikz}
\usepackage[plain]{algorithm}
\usepackage{algpseudocode}

\usetikzlibrary{automata,positioning}

%
% Basic Document Settings
%

\topmargin=-0.45in
\evensidemargin=0in
\oddsidemargin=0in
\textwidth=6.5in
\textheight=9.0in
\headsep=0.25in

\linespread{2.0}

\pagestyle{fancy}
\lhead{\hmwkAuthorName}
\chead{\hmwkClass\ (\hmwkClassInstructor\ \hmwkClassTime)}
\rhead{\firstxmark}
\lfoot{\lastxmark}
\cfoot{\thepage}

\renewcommand\headrulewidth{0.4pt}
\renewcommand\footrulewidth{0.4pt}

\setlength\parindent{0pt}

%
% Create Problem Sections
%

\newcommand{\enterProblemHeader}[1]{
    \nobreak\extramarks{}{Problem \arabic{#1} continued on next page\ldots}\nobreak{}
    \nobreak\extramarks{Problem \arabic{#1} (continued)}{Problem \arabic{#1} continued on next page\ldots}\nobreak{}
}

\newcommand{\exitProblemHeader}[1]{
    \nobreak\extramarks{Problem \arabic{#1} (continued)}{Problem \arabic{#1} continued on next page\ldots}\nobreak{}
    \stepcounter{#1}
    \nobreak\extramarks{Problem \arabic{#1}}{}\nobreak{}
}

\setcounter{secnumdepth}{0}
\newcounter{partCounter}
\newcounter{homeworkProblemCounter}
\setcounter{homeworkProblemCounter}{1}
\nobreak\extramarks{Problem \arabic{homeworkProblemCounter}}{}\nobreak{}

%
% Homework Problem Environment
%
% This environment takes an optional argument. When given, it will adjust the
% problem counter. This is useful for when the problems given for your
% assignment aren't sequential. See the last 3 problems of this template for an
% example.
%
\newenvironment{homeworkProblem}[1][-1]{
    \ifnum#1>0
        \setcounter{homeworkProblemCounter}{#1}
    \fi
    \section{Problem \arabic{homeworkProblemCounter}}
    \setcounter{partCounter}{1}
    \enterProblemHeader{homeworkProblemCounter}
}{
    \exitProblemHeader{homeworkProblemCounter}
}

%
% Homework Details
%   - Title
%   - Due date
%   - Class
%   - Section/Time
%   - Instructor
%   - Author
%

\newcommand{\hmwkTitle}{Homework\ \#9}
\newcommand{\hmwkDueDate}{November 6th, 2015}
\newcommand{\hmwkClass}{Differential Equation}
\newcommand{\hmwkClassTime}{Section 061}
\newcommand{\hmwkClassInstructor}{Professor Heather Lee}
\newcommand{\hmwkAuthorName}{Yao Xiao}
\newcommand{\La}{\mathcal{L}}

%
% Title Page
%

\title{
    \vspace{2in}
    \textmd{\textbf{\hmwkClass:\ \hmwkTitle}}\\
    \normalsize\vspace{0.1in}\small{Due\ on\ \hmwkDueDate\ at 3:10pm}\\
    \vspace{0.1in}\large{\textit{\hmwkClassInstructor\ \hmwkClassTime}}
    \vspace{3in}
}

\author{\textbf{\hmwkAuthorName}}
\date{}

\renewcommand{\part}[1]{\textbf{\large Part \Alph{partCounter}}\stepcounter{partCounter}\\}

%
% Various Helper Commands
%

% Useful for algorithms
\newcommand{\alg}[1]{\textsc{\bfseries \footnotesize #1}}

% For derivatives
\newcommand{\deriv}[1]{\frac{\mathrm{d}}{\mathrm{d}x} (#1)}

% For partial derivatives
\newcommand{\pderiv}[2]{\frac{\partial}{\partial #1} (#2)}

% Integral dx
\newcommand{\dx}{\mathrm{d}x}

% Alias for the Solution section header
\newcommand{\solution}{\textbf{\large Solution}}

% Probability commands: Expectation, Variance, Covariance, Bias
\newcommand{\E}{\mathrm{E}}
\newcommand{\Var}{\mathrm{Var}}
\newcommand{\Cov}{\mathrm{Cov}}
\newcommand{\Bias}{\mathrm{Bias}}

\begin{document}

\maketitle

\pagebreak


\begin{homeworkProblem}
\subsection{6.1-5b}
From the table, we get
\[
\begin{split}
f(t)=t^2\\
\La(t^2)=\frac{2!}{s^3} \\
\end{split}
\]
\end{homeworkProblem}

\begin{homeworkProblem}
\subsection{6.1-8}

From the table, we get
\[
\La(sinh at)=\frac{a}{s^2-a^2} \\\]\[
\La(sinh bt)=\frac{b}{s^2-b^2}
\]
\end{homeworkProblem}

\begin{homeworkProblem}
\subsection{6.1-15}
From the table, we get
\[ \La(t^ne^{at})=\frac{n!}{(s-a)^{n+1}}  \]
So
\[ \La(te^{at})=\frac{1}{(s-a)^{2}}  \]


\end{homeworkProblem}
\begin{homeworkProblem}

\subsection{6.2-10}
\[
\begin{split}
F(s)&=\frac{2s-3}{s^2+2s+10}\\
&=\frac{2s-3}{(s+1)^2+9}\\
&=\frac{2(s+1)-5}{(s+1)^2+9}\\
&= 2[\frac{s+1}{(s+1)^2+3}]-\frac{5}{3}[\frac{3}{(s+1)^2+3^2}]
\end{split}
\]
Looking up the table, plug it in, we get.

\[
\La^{-1}(F(s))=2e^{-t}cos(3t)-\frac{5}{3}e^{-t}sin(3t)
\]
\end{homeworkProblem}


\begin{homeworkProblem}
\subsection{6.2-21}
\[
\begin{split}
	y''-2y'+2y=cos(t)\\
	y(0)=1\\
	y'(0)=0\\
\La(y'')-2\La(y')+2\La(y)=\La(cost)\\
\La(y'')=s^2\La(y)-s\\
\La(y')=s\La(y)-1 \\
\La(cos(t))=\frac{s}{s^2+1} \\
\end{split}
\]

Plug it in, we get

\[
	\La(y)=\frac{s^3-2s^2+2s-2}{(s^2-2s+2)(s^2+1)}
\]
We dispose the equation above, we could get

\[
	\La(y)=\frac{4}{5}\frac{s-1}{(s-1)^2+1}-\frac{2}{5}	[\frac{1}{(s-1)^2+1}]+1/5\frac{s}{s^2+1}-2/5\frac{1}{s^2+1}
\]

Plug it in with the table, we get the solution

\[
y(t)=\frac{4}{5}e^tcost-\frac{2}{5}e^tsin(t)+\frac{1}{5	}cos(t)-\frac{2}{5}sin(t)
\]
\end{homeworkProblem}

\begin{homeworkProblem}
\subsection{6.3-1}
\[
g(t) =
\begin{cases} 
0 &  0<=t<1 \\ 
1 &  1<=t<3 \\ 
3 &  3<=t<4 \\ 
-3 &  4<=t<\infty \\ 

\end{cases}
\] 

\end{homeworkProblem}

\begin{homeworkProblem}
\subsection{6.3-2}

\[
g(t) =
\begin{cases} 
0 &  0<=t<2 \\ 
t-3 &  2<=t<3 \\ 
-1 &  3<=t<\infty \\ 
\end{cases}
\] 

\end{homeworkProblem}

\begin{homeworkProblem}
\subsection{6.3-18}
\[
\begin{split}
f(t)&=t-u_1(t)(t-1)\\
\La(f(t))&=\La(t)-\La(t-1)u_1(t)\\
&=\frac{1}{s^2}-e^{-s}\frac{1}{s^2}\\
&=\frac{(1-e^{-s})}{s^2}
\end{split}
\]
\end{homeworkProblem}

\begin{homeworkProblem}
\subsection{6.3-21}
\[
\begin{split}
F(s)&=\frac{2(s-1)e^{-2s}}{s^2-2s+2}\\
&=2\La(\frac{(s-1)e^{-2s}}{(s-1)^2+1}\\
\end{split}
\]
So we could transfer it back using the formula on the table. We get

\[
\La^{-1}(F(s))=2u_2(t)e^{t-2}cos(t-2)
\]
\end{homeworkProblem}

\begin{homeworkProblem}
\subsection{6.3-23}
\[
F(s)=\frac{(s-2)e^{-s}}{(s-2)^2-1}
\]
Let \[
g(s)=\frac{s-2}{(s-2)^2-1}
\]
Since \[
\La(e^{2t}cosht=\frac{s-2}{(s-2)^2-1}
\]
So the result is going to be

\[
\La^{-1}(F(s))=\La^{-1}(e^{-s}g(s))=f(t-1)u(t-1)
\]

which is equal to
\[
e^{2t-2}cosh(t-1)u(t-1)
\]
\end{homeworkProblem}


\end{document}
