\documentclass{article}

\usepackage{fancyhdr}
\usepackage{extramarks}
\usepackage{amsmath}
\usepackage{amsthm}
\usepackage{amsfonts}
\usepackage{tikz}
\usepackage[plain]{algorithm}
\usepackage{algpseudocode}

\usetikzlibrary{automata,positioning}

%
% Basic Document Settings
%

\topmargin=-0.45in
\evensidemargin=0in
\oddsidemargin=0in
\textwidth=6.5in
\textheight=9.0in
\headsep=0.25in

\linespread{1.1}

\pagestyle{fancy}
\lhead{\hmwkAuthorName}
\chead{\hmwkClass\ (\hmwkClassInstructor\ \hmwkClassTime): \hmwkTitle}
\rhead{\firstxmark}
\lfoot{\lastxmark}
\cfoot{\thepage}

\renewcommand\headrulewidth{0.4pt}
\renewcommand\footrulewidth{0.4pt}

\setlength\parindent{0pt}

%
% Create Problem Sections
%

\newcommand{\enterProblemHeader}[1]{
    \nobreak\extramarks{}{Problem \arabic{#1} continued on next page\ldots}\nobreak{}
    \nobreak\extramarks{Problem \arabic{#1} (continued)}{Problem \arabic{#1} continued on next page\ldots}\nobreak{}
}

\newcommand{\exitProblemHeader}[1]{
    \nobreak\extramarks{Problem \arabic{#1} (continued)}{Problem \arabic{#1} continued on next page\ldots}\nobreak{}
    \stepcounter{#1}
    \nobreak\extramarks{Problem \arabic{#1}}{}\nobreak{}
}

\setcounter{secnumdepth}{0}
\newcounter{partCounter}
\newcounter{homeworkProblemCounter}
\setcounter{homeworkProblemCounter}{1}
\nobreak\extramarks{Problem \arabic{homeworkProblemCounter}}{}\nobreak{}

%
% Homework Problem Environment
%
% This environment takes an optional argument. When given, it will adjust the
% problem counter. This is useful for when the problems given for your
% assignment aren't sequential. See the last 3 problems of this template for an
% example.
%
\newenvironment{homeworkProblem}[1][-1]{
    \ifnum#1>0
        \setcounter{homeworkProblemCounter}{#1}
    \fi
    \section{Problem \arabic{homeworkProblemCounter}}
    \setcounter{partCounter}{1}
    \enterProblemHeader{homeworkProblemCounter}
}{
    \exitProblemHeader{homeworkProblemCounter}
}

%
% Homework Details
%   - Title
%   - Due date
%   - Class
%   - Section/Time
%   - Instructor
%   - Author
%

\newcommand{\hmwkTitle}{Homework\ \#2}
\newcommand{\hmwkDueDate}{January 24th, 2017}
\newcommand{\hmwkClass}{Elements of Algebra I}
\newcommand{\hmwkClassTime}{Section 161}
\newcommand{\hmwkClassInstructor}{Professor Deepam Patel}
\newcommand{\hmwkAuthorName}{Yao Xiao}

%
% Title Page
%

\title{
    \vspace{2in}
    \textmd{\textbf{\hmwkClass:\ \hmwkTitle}}\\
    \normalsize\vspace{0.1in}\small{Due\ on\ \hmwkDueDate\  }\\
    \vspace{0.1in}\large{\textit{\hmwkClassInstructor\ \hmwkClassTime}}
    \vspace{3in}
}

\author{\textbf{\hmwkAuthorName}}
\date{}

\renewcommand{\part}[1]{\textbf{\large Part \Alph{partCounter}}\stepcounter{partCounter}\\}

%
% Various Helper Commands
%

% Useful for algorithms
\newcommand{\alg}[1]{\textsc{\bfseries \footnotesize #1}}

% For derivatives
\newcommand{\deriv}[1]{\frac{\mathrm{d}}{\mathrm{d}x} (#1)}

% For partial derivatives
\newcommand{\pderiv}[2]{\frac{\partial}{\partial #1} (#2)}

% Integral dx
\newcommand{\dx}{\mathrm{d}x}

% Alias for the Solution section header
\newcommand{\solution}{\textbf{\large Solution}}

% Probability commands: Expectation, Variance, Covariance, Bias
\newcommand{\E}{\mathrm{E}}
\newcommand{\Var}{\mathrm{Var}}
\newcommand{\Cov}{\mathrm{Cov}}
\newcommand{\Bias}{\mathrm{Bias}}

\begin{document}

\maketitle

\pagebreak

\begin{homeworkProblem}
Proof by induction \\
When n=1,
\[
\begin{split}
1=\frac{n(n+1)(2n+1)}{6}=\frac{1*2*3}{6}
\end{split}
\]
Suppose n=k
\[
\begin{split}
1+2^2+3^2+...+k^2=\frac{k(k+1)(2k+1)}{6}\\
1+2^2+3^2+...+k^2+(k+1)^2&=\frac{k(k+1)(2k+1)}{6}+(k+1)^2\\
&=\frac{6(k+1)^2+k(k+1)(2k+1)}{6}\\
&=\frac{(k+1)(6(k+1)+(2k+1)k)}{6}\\
&=\frac{(k+1)(6k+6+2k^2+k)}{6}\\
&=\frac{(k+1)(2k^2+7k+6)}{6}\\
&=\frac{(k+1)(2k+3)(k+2)}{6}\\
&=\frac{(k+1)(2(k+1)+1)((k+1)+1)}{6}\\
\end{split}
\]
So the equation holds for n=k+1. Therefore, the inequalities hold for all positive integers n > 1 by induction

\end{homeworkProblem}

\begin{homeworkProblem}
Proof by induction \\
When n=2 \\
We have \(1<8/3<5\), so the inequalities hold when n=2

When n is greater than 2, \[
\begin{split}
1+2^2+3^2+...+(n-1)^2+n^2&<\frac{n^3}{3}+n^2\\
&=\frac{n^3+3n^2}{3}\\
&<\frac{n^3+3n^2+3n+1}{3}\\
&=\frac{(n+1)^3}{3}
\end{split}
\]
So the left inequality holds
\[
\begin{split}
1+2^2+3^2+...n^2+(n+1)^2&>\frac{n^3}{3}+(n+1)^2\\
&=\frac{n^3+3(n+1)^2}{3}\\
&=\frac{n^3+3n^2+6n+3}{3}\\
&>\frac{n^3+3n^2+3n+1}{3}\\
&=\frac{(n+1)^3}{3}
\end{split}
\]
So the right inequality holds as well
\end{homeworkProblem}
\begin{homeworkProblem}
\begin{enumerate}
\item Let a be the integer, (a-a)=0. 0 mod anything is 0. So it holds
\item Let a and b be integers. Suppose that a $\equiv$ b (mod n), aka a-b = kn. b-a=-(a-b)=(-k)n. So it holds
\item Let a,b,c be integers. a-b=nk, b-c=nj, let the first one plus the second one a-b+b-c=nk+nj, a-c=n(k+j) so a-c=nk' as well
\end{enumerate}
\end{homeworkProblem}
\begin{homeworkProblem}
\( a \rightarrow b \) is true iff \( \neg b \rightarrow \neg a \) is true.
So we assume a is even, and the product of the two even number is \(2k*2k=2(2k)\) and it's even too.
\end{homeworkProblem}
\begin{homeworkProblem}
Let rotate be function R, so that \[R(1)=2\ R(2)=3\ R(3)=4\ R(4)=1 \]
Let flip be function F, so that \[F(1)=2\ F(2)=1\ F(3)=4\ F(4)=3 \]
When you rotate the square once, twice and three times. It will become (2,3,4,1),(3,4,1,2),(4,1,2,3), respectively. So when you do \(R^3*R\). it will become (1,2,3,4) which is I. So any rotation could be one of the I,\(R,R^2,R^3\)
\end{homeworkProblem}
\begin{homeworkProblem}
Assume we are starting from (1,2,3,4)
\[
\begin{split}
F=(2,1,4,3)\\
FR=R(2,1,4,3)=(3,2,1,4)\\
FR^2=R^2(2,1,4,3)=(4,3,2,1)\\
FR^3=R^3(2,1,4,3)=(1,4,3,2)
\end{split}
\]
\end{homeworkProblem}
\begin{homeworkProblem}
Since \(F^2=I\) and \(R^4=I\), there will be all the symmetric flips. Also it's obvious that you rotate first and flip later
\[RF=F(2,3,4,1)=(3,2,1,4)=FR\]
\end{homeworkProblem}
\begin{homeworkProblem}
The inverse of R is \(R^3\)\\
The inverse of \(R^2\) is \(R^2\)\\
The inverse of \(R^3\) is \(R\)
\end{homeworkProblem}


\end{document}
