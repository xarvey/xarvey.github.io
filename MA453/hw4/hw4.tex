\documentclass{article}

\usepackage{fancyhdr}
\usepackage{extramarks}
\usepackage{amsmath}
\usepackage{amsthm}
\usepackage{amsfonts}
\usepackage{tikz}
\usepackage[plain]{algorithm}
\usepackage{algpseudocode}
\usepackage{amssymb}



\usetikzlibrary{automata,positioning}

%
% Basic Document Settings
%

\topmargin=-0.45in
\evensidemargin=0in
\oddsidemargin=0in
\textwidth=6.5in
\textheight=9.0in
\headsep=0.25in

\linespread{1.1}

\pagestyle{fancy}
\lhead{\hmwkAuthorName}
\chead{\hmwkClass\ (\hmwkClassInstructor\ \hmwkClassTime): \hmwkTitle}
\rhead{\firstxmark}
\lfoot{\lastxmark}
\cfoot{\thepage}

\renewcommand\headrulewidth{0.4pt}
\renewcommand\footrulewidth{0.4pt}

\setlength\parindent{0pt}

%
% Create Problem Sections
%

\newcommand{\enterProblemHeader}[1]{
    \nobreak\extramarks{}{Problem \arabic{#1} continued on next page\ldots}\nobreak{}
    \nobreak\extramarks{Problem \arabic{#1} (continued)}{Problem \arabic{#1} continued on next page\ldots}\nobreak{}
}

\newcommand{\exitProblemHeader}[1]{
    \nobreak\extramarks{Problem \arabic{#1} (continued)}{Problem \arabic{#1} continued on next page\ldots}\nobreak{}
    \stepcounter{#1}
    \nobreak\extramarks{Problem \arabic{#1}}{}\nobreak{}
}

\setcounter{secnumdepth}{0}
\newcounter{partCounter}
\newcounter{homeworkProblemCounter}
\setcounter{homeworkProblemCounter}{1}
\nobreak\extramarks{Problem \arabic{homeworkProblemCounter}}{}\nobreak{}

%
% Homework Problem Environment
%
% This environment takes an optional argument. When given, it will adjust the
% problem counter. This is useful for when the problems given for your
% assignment aren't sequential. See the last 3 problems of this template for an
% example.
%
\newenvironment{homeworkProblem}[1][-1]{
    \ifnum#1>0
        \setcounter{homeworkProblemCounter}{#1}
    \fi
    \section{Problem \arabic{homeworkProblemCounter}}
    \setcounter{partCounter}{1}
    \enterProblemHeader{homeworkProblemCounter}
}{
    \exitProblemHeader{homeworkProblemCounter}
}

%
% Homework Details
%   - Title
%   - Due date
%   - Class
%   - Section/Time
%   - Instructor
%   - Author
%

\newcommand{\hmwkTitle}{Homework\ \#4}
\newcommand{\hmwkDueDate}{February 7th, 2017}
\newcommand{\hmwkClass}{Elements of Algebra I}
\newcommand{\hmwkClassTime}{Section 161}
\newcommand{\hmwkClassInstructor}{Professor Deepam Patel}
\newcommand{\hmwkAuthorName}{Yao Xiao}

%
% Title Page
%

\title{
    \vspace{2in}
    \textmd{\textbf{\hmwkClass:\ \hmwkTitle}}\\
    \normalsize\vspace{0.1in}\small{Due\ on\ \hmwkDueDate\  }\\
    \vspace{0.1in}\large{\textit{\hmwkClassInstructor\ \hmwkClassTime}}
    \vspace{3in}
}

\author{\textbf{\hmwkAuthorName}}
\date{}

\renewcommand{\part}[1]{\textbf{\large Part \Alph{partCounter}}\stepcounter{partCounter}\\}

%
% Various Helper Commands
%

% Useful for algorithms
\newcommand{\alg}[1]{\textsc{\bfseries \footnotesize #1}}

% For derivatives
\newcommand{\deriv}[1]{\frac{\mathrm{d}}{\mathrm{d}x} (#1)}

% For partial derivatives
\newcommand{\pderiv}[2]{\frac{\partial}{\partial #1} (#2)}

% Integral dx
\newcommand{\dx}{\mathrm{d}x}

% Alias for the Solution section header
\newcommand{\solution}{\textbf{\large Solution}}

% Probability commands: Expectation, Variance, Covariance, Bias
\newcommand{\E}{\mathrm{E}}
\newcommand{\Var}{\mathrm{Var}}
\newcommand{\Cov}{\mathrm{Cov}}
\newcommand{\Bias}{\mathrm{Bias}}

\begin{document}

\maketitle

\pagebreak

\begin{homeworkProblem}
1.\\a) \(GL_n(\mathbb{R}) \) it has an inverse because it's an invertible matrix, and also the product of two real numbers is a real number so the product of two real numbers matrix is a real number matrix. The identity matrix is invertible as well\\
b) it's a sub group, since multiplication of 1,-1 is closed under \( \{1,-1\} \), and identity 1 \( \in \{1,-1\} \) , and also \(\{1^{-1},-1^{-1}\} \) is in \( \{1,-1\} \)\\
c) it's not a subgroup, since the inverse of positive integer is not positive
\end{homeworkProblem}

\begin{homeworkProblem}
a) \( Z_2 * Z_2 * Z_2 \), since there's no generator \\
b) \( Z_2 * Z_2 = \{  (0,1), (0,0), (1,0), (1,1) \} \) Since no elements to itself will equal to (1,1)
\end{homeworkProblem}

\begin{homeworkProblem}
For \(k \in \mathbb{Z}_8 \) , gcd(8, k) = 1, k = 1, 3, 5, 7
For \(k \in \mathbb{Z}_{20} \) , gcd(20, k) = 1,  k = 1, 3, 7, 9, 11, 13, 17, 19
\end{homeworkProblem}

\begin{homeworkProblem}
Let the order of ab be n, so
 \[\begin{split}
(ab)^n=e\\
 (ab)(ab)(ab)...(ab)=e\\
 a(ba)(ba)(ba)(ba)....(ba)b=e\\
 a^{-1}a(ba)(ba)(ba)(ba)....(ba)ba=e\\
 (ba)^n=e 
 \end{split}\]
 So ba has an order of n too.
\end{homeworkProblem}

\begin{homeworkProblem}
We must show that g is close. So assume 
 \[\begin{split}
a,b \in Z, g \in G, g^{-1} \in G\\
ag=ga\\
bg=gb\\
ag^{-1}=g^{-1}a\\
bg^{-1}=g^{-1}b\\
abgg^{-1}=agbg^{-1}=agg^{-1}b=gag^{-1}b=gg^{-1}ab
 \end{split}\]
 so ab is close under Z.
Let \( a \in Z\) n. a has an
inverse \(a^{-1} \in G \)  \\
we want to show that \(a^{-1} \in Z\) \\
let \( g, g^{-1} \in G\)
 \[\begin{split}
(g^{-1}a^{-1}g)^{-1}=g^{-1}ag=g^{-1}ga=a
 \end{split}\]
 Which implies \(g^{-1}a^{-1}g=a^{-1}\), multiply \(g^{-1}\) on the both sides, we get  \(g^{-1}a^{-1}=a^{-1}g^{-1}\). So \(a^{-1} \in Z\)

\end{homeworkProblem}

\begin{homeworkProblem}
\[\begin{split}
|g^k|=\frac{n}{gcd(n,k)}\\
gcd(40,k)=40/10=4\\
 \end{split}\]
So all elements in Z40 has a gcd of 4 is the answer, which is 4,12,28,36. Respectively, \(x^4,x^{12},x^{28},x^{36}\) are the elements of order 10.

\end{homeworkProblem}

\begin{homeworkProblem}
We could show that kr=n. So a cyclic group of order n is equal to a group of \(\{e^0,e^1,e^2...e^{n-1} \} \). We could easily find a subgroup with an order r. \(\{e^0,e^k,e^(2k)...e^{k(r-1)} \} \) We assume there's another group which has an order of r, namely G'. And we could tell that G'=\(<k'>\) and k'r=n. But according to division theory, there exists only unique integers q,r such that a=qb+r , so k=k'.
\end{homeworkProblem}

\begin{homeworkProblem}
We assume that  \(d=gcd(n',n)>1\). Then \(k=\frac{n'n}{d}\) is an integer since n/d is an integer, as a result every element of \(Z_n * Z_m\) has an order dividing k, and it can't be cyclic because of that.

When  \(d=gcd(n',n)=1\), then  \(k=\frac{n'n}{d}=nn'\), so \(<1,1>\) could be the generator of the group so it's cyclic.
\end{homeworkProblem}



\end{document}
