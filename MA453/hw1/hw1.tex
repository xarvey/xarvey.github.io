\documentclass{article}

\usepackage{fancyhdr}
\usepackage{extramarks}
\usepackage{amsmath}
\usepackage{amsthm}
\usepackage{amsfonts}
\usepackage{tikz}
\usepackage[plain]{algorithm}
\usepackage{algpseudocode}

\usetikzlibrary{automata,positioning}

%
% Basic Document Settings
%

\topmargin=-0.45in
\evensidemargin=0in
\oddsidemargin=0in
\textwidth=6.5in
\textheight=9.0in
\headsep=0.25in

\linespread{1.1}

\pagestyle{fancy}
\lhead{\hmwkAuthorName}
\chead{\hmwkClass\ (\hmwkClassInstructor\ \hmwkClassTime): \hmwkTitle}
\rhead{\firstxmark}
\lfoot{\lastxmark}
\cfoot{\thepage}

\renewcommand\headrulewidth{0.4pt}
\renewcommand\footrulewidth{0.4pt}

\setlength\parindent{0pt}

%
% Create Problem Sections
%

\newcommand{\enterProblemHeader}[1]{
    \nobreak\extramarks{}{Problem \arabic{#1} continued on next page\ldots}\nobreak{}
    \nobreak\extramarks{Problem \arabic{#1} (continued)}{Problem \arabic{#1} continued on next page\ldots}\nobreak{}
}

\newcommand{\exitProblemHeader}[1]{
    \nobreak\extramarks{Problem \arabic{#1} (continued)}{Problem \arabic{#1} continued on next page\ldots}\nobreak{}
    \stepcounter{#1}
    \nobreak\extramarks{Problem \arabic{#1}}{}\nobreak{}
}

\setcounter{secnumdepth}{0}
\newcounter{partCounter}
\newcounter{homeworkProblemCounter}
\setcounter{homeworkProblemCounter}{1}
\nobreak\extramarks{Problem \arabic{homeworkProblemCounter}}{}\nobreak{}

%
% Homework Problem Environment
%
% This environment takes an optional argument. When given, it will adjust the
% problem counter. This is useful for when the problems given for your
% assignment aren't sequential. See the last 3 problems of this template for an
% example.
%
\newenvironment{homeworkProblem}[1][-1]{
    \ifnum#1>0
        \setcounter{homeworkProblemCounter}{#1}
    \fi
    \section{Problem \arabic{homeworkProblemCounter}}
    \setcounter{partCounter}{1}
    \enterProblemHeader{homeworkProblemCounter}
}{
    \exitProblemHeader{homeworkProblemCounter}
}

%
% Homework Details
%   - Title
%   - Due date
%   - Class
%   - Section/Time
%   - Instructor
%   - Author
%

\newcommand{\hmwkTitle}{Homework\ \#1}
\newcommand{\hmwkDueDate}{January 17th, 2017}
\newcommand{\hmwkClass}{Elements of Algebra I}
\newcommand{\hmwkClassTime}{Section 161}
\newcommand{\hmwkClassInstructor}{Professor Deepam Patel}
\newcommand{\hmwkAuthorName}{Yao Xiao}

%
% Title Page
%

\title{
    \vspace{2in}
    \textmd{\textbf{\hmwkClass:\ \hmwkTitle}}\\
    \normalsize\vspace{0.1in}\small{Due\ on\ \hmwkDueDate\  }\\
    \vspace{0.1in}\large{\textit{\hmwkClassInstructor\ \hmwkClassTime}}
    \vspace{3in}
}

\author{\textbf{\hmwkAuthorName}}
\date{}

\renewcommand{\part}[1]{\textbf{\large Part \Alph{partCounter}}\stepcounter{partCounter}\\}

%
% Various Helper Commands
%

% Useful for algorithms
\newcommand{\alg}[1]{\textsc{\bfseries \footnotesize #1}}

% For derivatives
\newcommand{\deriv}[1]{\frac{\mathrm{d}}{\mathrm{d}x} (#1)}

% For partial derivatives
\newcommand{\pderiv}[2]{\frac{\partial}{\partial #1} (#2)}

% Integral dx
\newcommand{\dx}{\mathrm{d}x}

% Alias for the Solution section header
\newcommand{\solution}{\textbf{\large Solution}}

% Probability commands: Expectation, Variance, Covariance, Bias
\newcommand{\E}{\mathrm{E}}
\newcommand{\Var}{\mathrm{Var}}
\newcommand{\Cov}{\mathrm{Cov}}
\newcommand{\Bias}{\mathrm{Bias}}

\begin{document}

\maketitle

\pagebreak

\newpage
\newpage

\begin{homeworkProblem}
1. The elements of the set are all numbers between 1 and 3, excluding 1 and 3 \\
2. An empty set
\end{homeworkProblem}

\begin{homeworkProblem}
\[	
\begin{split}
A \cap (B \setminus C) &:= \{x: x \in A \ and (\ x \in B \ and \ x \not \in C) \} \\
&= \{x: x \in A \ and  \ \ x \in B \ and \ x \not \in C) \} \\
&= \{x: (x \in A \ and  \ \ x \in B ) \ and \ x \not \in C) \}\\
&= \{x: x \in A \cap B \ and \ x \not \in C) \}\\
&= (A \cap B) \setminus C
	\end{split}
	 \]
\end{homeworkProblem}

\begin{homeworkProblem}
\[	
\begin{split}
(A \cap B) ^c :&= \{x \in S \ and \  x \notin (A \cap B)\}\\
&= \{x \in S \ and \  x \notin A \ and \ x \notin B)\}\\
&= \{x \in S \ and \  x \notin A \}  \cap \{x \in S \ and \ x \notin B) \}\\
&= \{x \in S \ and \  x \in A^c \}  \cap \{x \in S \ and \ x \in B^c) \}\\
&= A^c \cap B^c
	\end{split}
	 \]
\end{homeworkProblem}

%4
\begin{homeworkProblem}
\[	
\begin{split}
(A \times B) \cap (C \times D) &= \{(x,y): x \in A \ and \ y \in B \}
\cap \{(x,y): x \in C \ and \ y \in D \}\\
&= \{(x,y): x \in A \ and \ y \in B \ and \ x \in C \ and \ y \in D \}\\
&= \{(x,y): x \in A \ and \ x \in C \ and \ y \in B \ and \ y \in D \}\\
&= \{(x,y): x \in A \cap C \ and \ y \in B \cap  D \}\\
&= (A \cap C) \times (B \cap D)
	\end{split}
	 \]
\end{homeworkProblem}

%5
\begin{homeworkProblem}
\[	
\begin{split}
LHS = (\neg (p \rightarrow q)) = \neg (\neg p \lor (p \land q))
=p \land \neg (p \land q) = p \land (\neg p \lor \neg q) =  (p \land \neg p) \lor (p \land \neg q)\\
 = False \lor  p \land \neg q = p \land \neg q = RHS
	\end{split}
	 \]
\end{homeworkProblem}

%6
\begin{homeworkProblem}
Suppose the two numbers are a and b
\[
\begin{split}
\exists x,y \in \mathbb{Z} \ a=2x+1, b=2y+1 \\ a+b=2x+2y+2=2(x+y+1) 
	\end{split}
\]
So the sum of the two odd number is even
\end{homeworkProblem}

%7
\begin{homeworkProblem}
Suppose x+4 is odd, then \( \exists y \ x+4=2y+1 \ x+7=x+4+3=2y+1+3=2(y+2) \)\\
So x+7 is even.\\
On the other hand, suppose x+7 is even then \( \exists y \ x+7=2y \ x+4=x+7-3=2y-3=2(y-2)+1 \)\\
So x+4 is odd
\end{homeworkProblem}

%8
\begin{homeworkProblem}
\[
\begin{split}
A = \{1,2\}\\
B = \{2,3\}\\
A \setminus B = \{1\}\\
B \setminus A = \{3\}\\
\end{split}
\]
\end{homeworkProblem}

\end{document}
